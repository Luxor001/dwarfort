\documentclass{llncs}
%%%%%%%%%%%%%%%%%%%%%%%%%%%%%%%%%%%%%%%%%%%%%%%%%%%%%%%%%%%
%% package sillabazione italiana e uso lettere accentate
\usepackage[latin1]{inputenc}
\usepackage[english]{babel}
\usepackage[T1]{fontenc}
%%%%%%%%%%%%%%%%%%%%%%%%%%%%%%%%%%%%%%%%%%%%%%%%%%%%%%%%%%%%%

\usepackage{url}
\usepackage{xspace}

\makeatletter
%%%%%%%%%%%%%%%%%%%%%%%%%%%%%% User specified LaTeX commands.
\usepackage{manifest}

\makeatother


%%%%%%%
 \newif\ifpdf
 \ifx\pdfoutput\undefined
 \pdffalse % we are not running PDFLaTeX
 \else
 \pdfoutput=1 % we are running PDFLaTeX
 \pdftrue
 \fi
%%%%%%%
 \ifpdf
 \usepackage[pdftex]{graphicx}
 \else
 \usepackage{graphicx}
 \fi
%%%%%%%%%%%%%%%
 \ifpdf
 \DeclareGraphicsExtensions{.pdf, .jpg, .tif}
 \else
 \DeclareGraphicsExtensions{.eps, .jpg}
 \fi
%%%%%%%%%%%%%%%

\newcommand{\java}{\textsf{Java}}
\newcommand{\contact}{\emph{Contact}}
\newcommand{\corecl}{\texttt{corecl}}
\newcommand{\medcl}{\texttt{medcl}}
\newcommand{\msgcl}{\texttt{msgcl}}
\newcommand{\android}{\texttt{Android}}
\newcommand{\dsl}{\texttt{DSL}}
\newcommand{\jazz}{\texttt{Jazz}}
\newcommand{\rtc}{\texttt{RTC}}
\newcommand{\ide}{\texttt{Contact-ide}}
\newcommand{\xtext}{\texttt{XText}}
\newcommand{\xpand}{\texttt{Xpand}}
\newcommand{\xtend}{\texttt{Xtend}}
\newcommand{\pojo}{\texttt{POJO}}
\newcommand{\junit}{\texttt{JUnit}}

\newcommand{\action}[1]{\texttt{#1}\xspace}
\newcommand{\code}[1]{{\small{\texttt{#1}}}\xspace}
\newcommand{\codescript}[1]{{\scriptsize{\texttt{#1}}}\xspace}

% Cross-referencing
\newcommand{\labelsec}[1]{\label{sec:#1}}
\newcommand{\xs}[1]{\sectionname~\ref{sec:#1}}
\newcommand{\xsp}[1]{\sectionname~\ref{sec:#1} \onpagename~\pageref{sec:#1}}
\newcommand{\labelssec}[1]{\label{ssec:#1}}
\newcommand{\xss}[1]{\subsectionname~\ref{ssec:#1}}
\newcommand{\xssp}[1]{\subsectionname~\ref{ssec:#1} \onpagename~\pageref{ssec:#1}}
\newcommand{\labelsssec}[1]{\label{sssec:#1}}
\newcommand{\xsss}[1]{\subsectionname~\ref{sssec:#1}}
\newcommand{\xsssp}[1]{\subsectionname~\ref{sssec:#1} \onpagename~\pageref{sssec:#1}}
\newcommand{\labelfig}[1]{\label{fig:#1}}
\newcommand{\xf}[1]{\figurename~\ref{fig:#1}}
\newcommand{\xfp}[1]{\figurename~\ref{fig:#1} \onpagename~\pageref{fig:#1}}
\newcommand{\labeltab}[1]{\label{tab:#1}}
\newcommand{\xt}[1]{\tablename~\ref{tab:#1}}
\newcommand{\xtp}[1]{\tablename~\ref{tab:#1} \onpagename~\pageref{tab:#1}}
% Category Names
\newcommand{\sectionname}{Section}
\newcommand{\subsectionname}{Subsection}
\newcommand{\sectionsname}{Sections}
\newcommand{\subsectionsname}{Subsections}
\newcommand{\secname}{\sectionname}
\newcommand{\ssecname}{\subsectionname}
\newcommand{\secsname}{\sectionsname}
\newcommand{\ssecsname}{\subsectionsname}
\newcommand{\onpagename}{on page}

\newcommand{\xauthA}{Jun Hao Xia}
\newcommand{\xauthB}{Andrea Vecchiotti}
\newcommand{\xauthC}{Stefano Belli}
\newcommand{\xfaculty}{II Faculty of Engineering}
\newcommand{\xunibo}{Alma Mater Studiorum -- University of Bologna}
\newcommand{\xaddrBO}{viale Risorgimento 2}
\newcommand{\xaddrCE}{via Venezia 52}
\newcommand{\xcityBO}{40136 Bologna, Italy}
\newcommand{\xcityCE}{47023 Cesena, Italy}

%
% Comments
%
%%% \newcommand{\todo}[1]{\bf{TODO:}\emph{#1}}


\begin{document}

\title{Software Engineering\\
 process report template}

%%% \author{\xauthA,\xauthB and \xauthC}
\author{\xauthB, \xauthB, \xauthC}

\institute{%
%%%  \xunibo\\\xaddrCE, \xcityCE\\\email{\{nameA.studentA, nameB.studentB\}@studio.unibo.it}
  \xunibo\\\xaddrCE, \xcityCE\\\email{\{andrea.vecchiotti, junhao.xia, stefano.belli4\}@studio.unibo.it}
}

\maketitle

%% \begin{abstract}
%% \footnotesize
%%This a Latex template to be used for the reports of Software Engineering.
%%\keywords{Software engineering, managed software development, reports, ....}
%%\end{abstract}



%%% \sloppy

%===========================================================================
\section{Introduction}
\labelsec{intro}
This document describes the process adopted as a team to develop a Software System prototype, in respect with the user given functional requiments.\\
All the different steps of analysis, prototyping and actual implementation of the system will be reported in this document and then analized from an engineering point of view.\\
Moreover, the actual phases of software development will be studied from an analytic point of view: apart from give a valid functioning prototype, one of the main objectives of this work will be to provide a valid case study for the applicance of Software Engineering tecniques.
%===========================================================================

%===========================================================================
\section{Vision}
\labelsec{Vision}
%FIXME: Da sistemare bibliografia. ho messo il riferimento a glass in template2013.bbl
The development of software systems has undergone a significant\cite{Glass97} number of failures (even quite large) during the period 1970-1980, thanks to a series of concurrent causes:
\begin{itemize}
\item Bad (or entirely absent) requiments management.
\item Weak system architectures.
\item Many inconsistencies between requiments, project and final realization.
\item Entirely absence of unit tests. %FIXME: controllare se il termine "unit test" va bene
\end{itemize}
Nowadays the situation has been (partially) solved by making software engineering principles one of the key study subjects in Computer Science: the adoption of good concepts of \textit{software creation and testing automation} has led to a better quality of software developed.\\\\
The main purpose of this project (final assignment of \textit{Ingengeria dei Sistemi Software}, lead by prof. \textit{Antonio Natali}) is to actually learn and succesfully adopt good concept of \textit{software creation and testing automation}, in order to develop more reliable products, both for the client and the developer.
\newpage
In fact, qRobot can serve as an excellent case study for:
\begin{itemize}
\item Understanding the role and the importance of \textit{Requiments analysis} and its pratical role on software production.\\
In fact, Software deployment cannot start without requiments from the clients, and any ambiguity of it must be cleared.\\
More generally, this statement can be considered useful amongst any non-trivial software development project:\\
{
\centering
There is no code without a project, no project without problem
analysis and no problem without requirements
}
\item aa
\end{itemize}
%tra le altre cose, dire di scrum: alla fine un obiettivo del progetto è utilizzarlo.

%L'idea principale è di essere sinceri: qual'è la vision di questo progetto? imparare ad adottare stili giusti di creazione software. è questa la vision


%As a team, we firmly believe that good software cannnot be achieved in a reasonable amount of time without following concepts of \textit{Agile development of software}; in fact, the usage of SCRUM framework for the development of this project has perfectly satisfied the needs of the team for collaborative work.\\
%As a matter of fact, each work iteration was organized by a done \textit{sprint planning}, aiding the team on scheduling the features for each \textit{Increment}; the set of requiments to accomplish on each iteration was led by the so-called \textit{sprint goal}, foundamental to share the same purpose in the team.\\\\%migliorare come forma? è corretto pienamente ciò che ho scritto?
%Inserire Analyze a little. Design a little. Code a little. Test what you can
%However, we also think that the adoption of concepts of \textit{Model Driven Software Development (MDSD)} can reduce %copiato spudoratamente da relazione2.pdf%
%time to market and improve the overall quality of the software: for this reason...\\\\%...il team ha adottato DSL in questo punti del progetto ecc.ecc.
%The main principle, amongst many others, that the team followed on the develop of this project is:
%\begin{center}
%\textit{There's no code without project, there's no project without problem analysis,
%there's no problem without requirements.}
%\end{center}
%The intepretation of this principle is pretty simple: without specifications and requiments analysis, it's impossible to talk about good development of software.\\
%In other terms, a detailed analysis and a study of traceability of requiments it's mandatory before writing any line of code; othewise, the risk of writing useless code it's highly probable.\\\\%scriverlo in una maniera più elegante..
%inserire frasi fatte del prof
%===========================================================================

%===========================================================================
\section{Goals}
\labelsec{Goals}

The robot should travel along a straight line from a point A set to a distance da in front of sonar1 to a point B in front of sonar2, at a distance approximarely equal to da. For example:\\
\includegraphics[scale = 0.5]{img/tf2017b.jpg}\\ \\
The starting point A is set by the final users at any distance 0<da<dm. The software controlling the robot must be able to detect fixed and mobile obstacles and select some alternative path to reach its destination. For example:\\
\includegraphics[scale = 0.3]{img/tf2017c.jpg}\includegraphics[scale = 0.3]{img/tf2017d.jpg}
%===========================================================================

%===========================================================================
\section{Requirements}
\labelsec{Requirements}

Design and build a (prototype of a) software system that:
\begin{itemize}
\item starts the robot (already put in A) when the user sends to it a start command by using a remote console;
\item drives the robot along a straight path from A to B if no onstacle is detected;
\item avoids the mobile obstacle by waiting (when detected) that the obstacle disappears from the robot's front;
\item avoids the fixed obstacles by finding some alternative path to reach its goal;
\item stops the robot if an alarm event occurs.
\end{itemize}

After completion of the first step, extend the system built at step1 so that a human user can take the control at any time after robot starting. In this case:
\begin{itemize}
\item the robot must stop its journey and execute movement commands sent by the user via the console. Obviously the robot should avoid to hit obstacles;
\item among these commands, the user can tell the robot to return in autonomous way to its starting point along the same path already covered, by avoiding to hit (mobile) obstacles.
\end{itemize}

%===========================================================================

 
%===========================================================================
\section{Requirement analysis}
\labelsec{ReqAnalysis}
%===========================================================================
\subsection{Use cases}
\labelssec{UseCases}

\subsection{Scenarios}
\labelssec{Scenarios}

\subsection{(Domain)model}

\subsection{Test plan}

%===========================================================================
\section{Problem analysis}
\labelsec{ProblemAnalysis}
%===========================================================================
\subsection{Logic architecture}
\subsection{Abstraction gap}
\subsection{Risk analysis}

%===========================================================================
\section{Work plan}
\labelsec{wplan}
%===========================================================================

%===========================================================================
\section{Project}
\labelsec{Project}
%===========================================================================

\subsection{Structure}
\subsection{Interaction}
\subsection{Behavior}

%===========================================================================
\section{Implementation}
\labelsec{Implementation}
%===========================================================================

%===========================================================================
\section{Testing}
\labelsec{testing}
%===========================================================================

%===========================================================================
\section{Deployment}
\labelsec{Deployment}
%===========================================================================

%===========================================================================
\section{Maintenance}
\labelsec{Maintenance}
%===========================================================================
\newpage
See \cite{natMol09} until page 11 (\texttt{CMM}) and pages 96-105.

%===========================================================================
\section{Information about the author}
\labelsec{Author}
%===========================================================================

\vskip.5cm
%%% \begin{figure}
\begin{tabular}{ | c |  }
\hline
  % after \\: \hline or \cline{col1-col2} \cline{col3-col4} ...
  Photo of the author 
  \\
\hline
   \includegraphics[scale = 0.7]{img/foto_autore.jpg}
  \\
\hline
\end{tabular}

% wtf che è sta roba?
%
%%% \begin{itemize}
%%% \item Titolo di studio:\\ \\
%%% \item Interessi particolari:\\ \\
%%% \item Ha sostenuto fino ad oggi il seguente numero di esami:\\ \\
%%% \item Deve ancora sostenere i seguenti esami del I anno:\\ \\
%%% \item Prevede di svolgere un tirocinio presso:\\ \\
%%% \item Prevede di laurearsi nella sessione:\\ \\
%%% \item Intende proseguire gli studi per conseguire: \\  \\  \\
%%%   	presso la sede universitaria di: \\ \\
%%% \item Intende entrare subito nel mondo del lavoro presso : \\ \\
%%% \end{itemize}

 
\appendix


\bibliographystyle{abbrv}
\bibliography{biblio}

\end{document}

%RELAZIONE (domande e concetti)
premessa da piaga: chiedo scusa, alla fine di ogni capitolo nei commenti,
aggiungo dei concetti di teoria per comodità e per orientarci sul cosa scrivere nella relazione


1)INTRODUZIONE
Scopo della relazione
Introduzione a metodologie usate per sviluppare il progetto
Scopo dell'elaborato

2)VISIONE (si faccia riferimento soprattutto alla visione del percorso di riferimento)
Concetti e punti di vista sulle metodologie usate
Concetto teorico su cui si basa lo sviluppo del sistema software
Perchè è necessaria l'analisi?
Quali metodologie e modelli si possono sfruttare?
Perchè non UML?
Quale modello si può usare per creare il codice autogenerato?
Definire gli step del processo di produzione

Un percorso di riferimento:
L’artefatto relativo alla visione può essere un documento
che descrive le motivazioni generali che hanno indotto
una persona,gruppo, azienda a impegnarsi nello sviluppo dell’applicazione,
evidenziandone i benefici che provengono da essa.
L’artefatto relativo agli obiettivi è un documento che illustra gli obiettivi
che si vogliono raggiungere grazie allo sviluppo dell’applicazione o al suo uso
o sviluppi futuri. 
Insieme agli obiettivi è possibile legare un modello di business,
per esprimere il contesto dove il sistema opera e per definire un
vocabolario di termini utili alla fase di analisi dei requisiti.

Commento by piaga:
é possibile scrivere la relazione mantenendo la terza persona?
Secondo i concetti della vision scrivere "the usage of SCRUM framework for the dev...has perfectly satisfied..." potrebbe non essere corretto perchè lo fa sembrare un parere sulla conclusione del progetto
rispetto ad una possibile visione sul modo di procedere.
IMHO consiglierei scrivere qualcosa tipo:

Scrum è un framework per lo sviluppo..bla bla bla (primissimi concetti di teoria)
Scrum può essere utile nello sviluppo del progetto (intenzioni)
Queste due frasi le metterei alla fine e non all'inizio...
prima scriverei molto sulla visione generale del progetto.
Non so se è giusto ma scriverei qualcosa tipo:
"(ambiente attuale)Nel mondo dell'informatizzazione e dell'automazione, sempre più professioni scompariranno... 
(previsione)...è giusto quindi prevedere la necessità di automatizzare anche la mobilità di un drone.
(benefici del progetto) l'intenzione è quello di minimizzare l'intervento umano sull'autonomia di lavoro del drone"





3)GOAL
Definire gli obiettivi di progetto e in che modo s'intende procedere.

4)REQUISITI
Scrivere i requisiti (copiare di pari passo ai requisiti del prof !?)

5)ANALISI DEI REQUISITI
Definire il glossario
Definire i casi d'uso
Definire il dominio-modello

ANALISI DEL PROBLEMA
Definire la metodologia con cui si effettua l'analisi del problema
Illustrare il problema di mantenere un buon livello di astrazione; da che piattaforma si parte, che linguaggio, in che modo si astrae? (fare riferimento alle relazioni)
Descrivere l'architettura logica
Descrivere il piano di testing

PIANO DI LAVORO
Come si vuole affrontare il problema dell'astrazione?
Quali sono i tempi e costi di lavoro?

PROGETTO IMPLEMENTAZIONE TESTING DEPLOYMENT MAINTENANCE
Guarda la relazione